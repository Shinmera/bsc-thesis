\section{Related Work}
\subsection{\sectioncite{s4}}
\imagefigure{images/s4-graphs.pdf}{Graphs of the three tests used to evaluate the S4 platform. From left to right: word count, click-through rate, and online parameter optimisation. Orange nodes are stateful. Nodes coloured in red retain state of previous stream events.}

\subsection{\sectioncite{spade}}
\imagefigure{images/spade-graphs.pdf}{Graph of the example used in the SPADE paper: a bargain index computation. Nodes coloured in red retain state of previous stream events.}

\subsection{\sectioncite{discretized}}
\imagefigure{images/discretized-graphs.pdf}{Graph of the Word Count example used to illustrate the discretized streams. Not shown is the incremental reduction done between batches.}

\subsection{\sectioncite{millwheel}}

\subsection{\sectioncite{bigdatabench}}

\subsection{\sectioncite{streamcloud}}
\imagefigure{images/streamcloud-graphs.pdf}{Graph of the query used to evaluate StreamCloud. Nodes coloured in red retain state of previous stream events.}

\subsection{\sectioncite{integrating}}
\imagefigure{images/integrating-graphs.pdf}{Illustration of the queries for the Linear Road Benchmark (top) and the Top-K (bottom) tests used to evaluate their system. Orange nodes maintain internal state.}

\subsection{\sectioncite{timestream}}
\imagefigure{images/timestream-graphs.pdf}{The Distinct Count query used to evaluate the Timestream system. Nodes in orange have internal state.}

\subsection{\sectioncite{storm}}
\imagefigure{images/storm-graphs.pdf}{The topology graph used to evaluate the Storm schedulers. Orange nodes are stateful. ``Shuffle'' connections send the output event to a random destination node. ``Fields'' connections send the output event to a specific destination node based on a field of the event.}

\subsection{\sectioncite{storm2}}
\imagefigure{images/storm2-graphs.pdf}{An illustration of the topology used for the Twitter \& Bitly link trend analysis. Nodes coloured orange maintain internal state.}

\subsection{Comparison}
\begin{figure}[H]
  \centering
  {
    \scriptsize
    \hspace*{-1.5cm}
    \begin{tabularx}{0.85\pagewidth}{|X|X|X|X|X|X|X|X|X|}
      \hline
      Paper & Goal & Application & Dataflows & Operators & Workloads & Testbed & External Systems & Reproducibility \\
      \hline
    \end{tabularx}
  }
  \caption{Comparison of various features of the reference papers.}
\end{figure}

%%% Local Variables:
%%% mode: latex
%%% TeX-master: "../thesis"
%%% TeX-engine: luatex
%%% End:
