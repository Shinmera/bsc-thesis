\section{Strymon Benchmark Framework}\label{section:framework}
In order to factor out common operations and to be able to cleanly define the data flows of each test in the benchmarks we developed a new framework. This framework takes care of processing command line arguments, Timely setup, and statistical data collection. All that's necessary in order to implement a benchmark is the implementation of the \code{Benchmark} trait for the overall benchmark, the implementation of the \code{TestImpl} trait for each test in the benchmark, and the implementation of necessary data conversion routines.

\subsection{Benchmarks and Tests}
\begin{listing}[H]
\begin{minted}{rust}
pub trait Benchmark {
    fn name(&self) -> &str;
    fn generate_data(&self, config: &Config) -> Result<()>;
    fn tests(&self) -> Vec<Box<Test>>;
}
\end{minted}
  \caption{Definition of the Benchmark trait.}
  \label{lst:benchmark-trait}
\end{listing}

The \code{Benchmark} includes a method to generate input data files that can be used to feed into the system from disk at a later point, and a function to return all the tests that the benchmark defines. Note that it returns \code{Test} trait instances rather than \code{TestImpl}. This is done to avoid leaking associated trait types into surrounding code that doesn't need to know about it.

\begin{listing}[H]
\begin{minted}{rust}
pub trait Test : Sync+Send {
    fn name(&self) -> &str;
    fn run(&self, config: &Config, worker: &mut Root<Generic>)
        -> Result<Statistics>;
}
\end{minted}
  \caption{Definition of the Test trait.}
  \label{lst:test-trait}
\end{listing}

The \code{Test} trait includes only one important method, which runs the test on a given worker and returns collected statistical data of the run if it completed was successfully. The \code{Test} trait is automatically implemented for all types that implement the \code{TestImpl} trait, thus avoiding the need to manually implement the \code{Test} trait, and automatically erasing the associated types required in the \code{TestImpl} definition.

\begin{listing}[H]
\begin{minted}{rust}
pub trait TestImpl : Sync+Send {
    type D: Data;
    type DO: Data;
    type T: Timestamp;

    fn name(&self) -> &str;
    fn create_endpoints(&self, &Config, index: usize, workers: usize)
        -> Result<(Source<Self::T, Self::D>, Drain<Self::T, Self::DO>)>;
    fn construct_dataflow<'scope>(&self, &Config, stream: 
          &Stream<Child<'scope, Root<Generic>, Self::T>, Self::D>)
        -> Stream<Child<'scope, Root<Generic>, Self::T>, Self::DO>;
    fn run(&self, &Config, worker: &mut Root<Generic>)
        -> Result<Statistics>;
}
\end{minted}
  \caption{Definition of the TestImpl trait.}
  \label{lst:testimpl-trait}
\end{listing}

The \code{TestImpl} trait is the most complicated, but should not be hard to utilise either. The \code{run} method has a default implementation and as such typically does not need to be implemented. This implementation makes use of the associated \code{D}, \code{DO}, and \code{T} types in order to work regardless of the test's type requirements. The \code{create_endpoints} method is responsible for initialising the input and output of the data flow. This method is required to be implemented in order for the test to function properly, but should be very straight-forward.

\begin{listing}[H]
\begin{minted}{rust}
fn create_endpoints(&self, config: &Config, _: usize, _: usize)
   -> Result<(Source<Self::T, Self::D>, Drain<Self::T, Self::DO>)> {
    Ok((Source::from_config(config,
          Source::new(Box::new(NEXMarkGenerator::new(config))))?,
        Drain::from_config(config)?))
}
\end{minted}
  \caption{A sample definition of the \code{create_endpoints} method.}
  \label{lst:create-endpoints}
\end{listing}

Finally, the most important method is \code{construct_dataflow}. It is responsible for transforming the input data stream into an output data stream, implementing the data flow logic. The function bodies of the respective \code{construct_dataflow} implementations are shown in the Data Flows subsections of each benchmark. A minimal complete example of a test definition is shown in \autoref{lst:create-test}.

\begin{listing}[H]
\begin{minted}{rust}
struct MyTest{}

impl TestImpl for MyTest{
    type T = usize;
    type D = String;
    type DO = (String, usize);

    fn create_endpoints(&self, _:&Config, _: usize, _: usize)
        -> Result<(Source<Self::T, Self::D>, Drain<Self::T, Self::DO>)>{
        Ok(Source::from(vec!((0, vec!("a", "b").map(String::from)),
                             (1, vec!("b", "c").map(String::from)))),
           Drain::from(io::stdout()))
    }
    
    fn construct_dataflow<'scope>(&self, _:&Config, stream: 
          &Stream<Child<'scope, Root<Generic>, Self::T>, Self::D>)
        -> Stream<Child<'scope, Root<Generic>, Self::T>, Self::DO>{
        stream.rolling_count(|w| w.clone(), |w, c| (w, c))
    }
}
\end{minted}
  \caption{A complete sample test definition. The endpoints are configured to feed data from a predefined vector, and output results to the standard output stream. The dataflow performs a simple word count.}
  \label{lst:create-test}
\end{listing}

\subsection{Input and Output System}
In order to factor out the common problem of handling the data input and result output of the dataflows, we developed a simple ``endpoints'' system. At the forefront are the opaque \code{Source} and \code{Drain} instances that wrap an inner \code{EventSource} and \code{EventDrain} trait instance respectively. This allows us to pass the sources and drains around without the rest of the system having to know about their particular properties. \\

\begin{listing}[H]
\begin{minted}{rust}
pub struct Source<T, D>(Box<EventSource<T, D>>);
pub struct Drain<T, D>(Box<EventDrain<T, D>>);

pub trait EventSource<T, D> {
    fn next(&mut self) -> Result<(T, Vec<D>)>;
}

pub trait EventDrain<T, D> {
    fn next(&mut self, T, Vec<D>);
}
\end{minted}
  \caption{The core traits and structs of the endpoint system.}
  \label{lst:endpoints}
\end{listing}

We provide the following implementations of the \code{EventSource} and \code{EventDrain} traits for ease of use:

\begin{itemize}
\item \code{Null} --- This struct does not provide any data as \code{EventSource}, and simply ignores all data as \code{EventDrain}.
\item \code{Console} --- As \code{EventSource}, this struct reads from the standard input stream and converts each line into input data. As \code{EventDrain} it simply writes each timestamp and data pair to the standard output.
\item \code{FileInput} --- This \code{EventSource} reads lines from a file and, similar to the \code{Console}, converts each line into a single input record.
\item \code{FileOutput} --- This \code{EventDrain} writes each record it is given to a file, one line per record.
\item \code{VectorEndpoint} --- This can be used both as \code{EventSource} and \code{EventDrain}, providing an in-memory stream for records.
\item \code{MeterOutput} --- This \code{EventDrain} simply reports how many records it sees for each epoch.
\end{itemize}

The \code{From} trait is implemented on both \code{Source} and \code{Drain} wherever applicable, such that constructing the proper endpoints is relatively trivial. \\

\begin{listing}[H]
\begin{minted}{rust}
pub struct Null{}

impl<T, D> EventSource<T, D> for Null {
    fn next(&mut self) -> Result<(T, Vec<D>)> {
        out_of_data()
    }
}

impl<T: Timestamp, D: Data> From<()> for Source<T, D> {
    fn from(_: ()) -> Source<T, D> {
        Source::new(Box::new(Null{}))
    }
}
\end{minted}
  \caption{The definition of the null event source.}
  \label{lst:null-endpoint}
\end{listing}

Some of these endpoints require the conversion of data to and from a serialised format such as strings. In order to facilitate this, the \code{ToData} and \code{FromData} traits need to be implemented by the user to perform the required conversion between their data types. An example of such conversion routines for the code from \autoref{lst:create-test} is provided in \autoref{lst:endpoint-conversion}.


\begin{listing}[H]
\begin{minted}{rust}
impl FromData<usize> for (String, usize) {
    fn from_data(&self, t: &usize) -> String {
        format!("{} {} ×{}", t, self.0, self.1)
    }
}
\end{minted}
  \caption{Record conversion to make MyTest work.}
  \label{lst:endpoint-conversion}
\end{listing}

\subsection{Additional Operators}
In order to ease the implementation of the benchmarks and improve the usability of the Timely system we introduced a number of additional operators. We will outline and discuss these operators here shortly.

\subsubsection{FilterMap}
This operator performs a filter followed by a map in one go. The operator expects a closure which can choose to either filter events by returning \code{None}, or map them by returning \code{Some(..)} with the output data.

\begin{listing}[H]
\begin{minted}{rust}
fn filter_map(&self, map) -> Stream {
    self.unary_stream(move |input, output| {
        input.for_each(|time, data| {
            let mut session = output.session(time);
            data.for_each(|x|{
                if let Some(d) = map(x) {
                    session.give(d);
                }
            });
        });
    })
}
\end{minted}
  \caption{Simplified code for the filter map operator.}
  \label{lst:filtermap}
\end{listing}

The operator is mostly useful for streams that contain multiple types of data encapsulated in an enum. In that case turning the stream into one of a single type is trivial using \code{filter_map}.

\begin{listing}[H]
\begin{minted}{rust}
enum Vehicle {
    Car(Car)
    Boat(Boat)
}

impl Car { 
    fn from(vehicle: Vehicle) -> Option<Car> {
        match vehicle {
            Vehicle::Car(car) => Some(car),
            _ => None
        }
    }
}

stream.filter_map(|x| Car::from(x))
\end{minted}
\caption{An example of the filter map operator to purify a stream of vehicles into one of cars.}
\label{lst:filtermap-example}
\end{listing}

\noindent This operator was used in \hyperref[section:nexmark]{NEXMark}.

\subsubsection{Join}
This operator offers two forms of joins that merge two separate streams of data into one. The first is an epoch based join, meaning data is only matched up between the two streams within a single epoch. If no match is found for either stream, the data is discarded.

\begin{listing}[H]
\begin{minted}{rust}
fn epoch_join(&self, stream, key_1, key_2, joiner) -> Stream{
    let mut epoch1 = HashMap::new();
    let mut epoch2 = HashMap::new();
    
    self.binary_notify(move |input1, input2, output, notificator| {
        // Gather all the records from the left side, group them into
        // vectors keyed according to the first key function, and
        // remember that vector for the current epoch.
        input1.for_each(|time, data|{
            let epoch = epoch1.entry(time).or_insert_with(HashMap::new);
            data.for_each(|dat|{
                let key = key_1(&dat);
                let datavec = epoch.entry(key).or_insert_with(Vec::new);
                datavec.push(dat);
            });
            notificator.notify_at(time);
        });
        // Perform the same but for the right side, using the second
        // key function.
        input2.for_each(|time, data|{
            let epoch = epoch2.entry(time).or_insert_with(HashMap::new);
            data.for_each(|dat|{
                let key = key_2(&dat);
                let datavec = epoch.entry(key).or_insert_with(Vec::new);
                datavec.push(dat);
            });
            notificator.notify_at(time);
        });
        // Once we notice an epoch completion we can join the two sides.
        notificator.for_each(|time, _, _|{
            if let Some(k1) = epoch1.remove(time) {
                if let Some(mut k2) = epoch2.remove(time) {
                    // With data from both sides available, we now do an
                    // inner join, duplicating data from the left side.
                    let mut out = output.session(time);
                    for (key, data1) in k1{
                        if let Some(mut data2) = k2.remove(&key) {
                            for d1 in data1 {
                                data2.for_each(|d2|
                                    out.give(joiner(d1.clone(), d2)));
                            }
                        }
                    }
                }
            } else {
                epoch2.remove(time);
            }
        });
    })
}
\end{minted}
  \caption{Simplified code for the epoch based join operator.}
  \label{lst:epoch-join}
\end{listing}

The epoch join is the more general join operator that should be useful whenever it is ensured that related events are emitted in the same epoch. If the out-of-orderness means the epochs could be different, a more general join operator is required.

\begin{listing}[H]
\begin{minted}{rust}
customers.epoch_join(coffees,
    |customer| customer.id, 
    |coffee| coffee.customer, 
    |customer, coffee| (customer.name, coffee.price))
\end{minted}
\caption{An example of an epoch join to determine how much each customer needs to pay for their coffee.}
\label{lst:epoch-join-example}
\end{listing}

The second form of join offered is a left join that keeps the left-hand stream's data around indefinitely, continuously joining it with data from the right-hand stream whenever the keys match.

\begin{listing}[H]
\begin{minted}{rust}
fn left_join(&self, stream, key_1, key_2, joiner) -> Stream{
    let mut d1s = HashMap::new();
    let mut d2s = HashMap::new();

    self.binary_notify(stream, move |input1, input2, output, _| {
        input1.for_each(|time, data|{
            data.for_each(|d1| {
                let k1 = key_1(&d1);
                if let Some(mut d2) = d2s.remove(&k1) {
                    output.session(time).give_iterator(
                        d2.map(|d| joiner(d1.clone(), d)));
                }
                d1s.insert(k1, d1);
            });
        });
        input2.for_each(|time, data|{
            data.for_each(|d2| {
                let k2 = key_2(&d2);
                if let Some(d1) = d1s.get(&k2) {
                    output.session(time).give(joiner(d1.clone(), d2));
                } else {
                    d2s.entry(k2).or_insert_with(Vec::new).push(d2);
                }
            });
        });
    })
}
\end{minted}
  \caption{Simplified code for the left join operator.}
  \label{lst:left-join}
\end{listing}

Left joins are useful when events on one side might only be emitted once, but events on the right hand side are recurring, meaning we need to retain the left side indefinitely. Such joins can also be implemented as co-FlatMaps using a static second input.

\begin{listing}[H]
\begin{minted}{rust}
driver_registrations.left_join(speeding_cars,
    |car| car.driver,
    |driver| driver.id,
    |car, driver| (car.license_plate, driver.address))
\end{minted}
\caption{An example of the left join operator, joining driver's registrations to cars that have been caught speeding.}
\label{lst:left-join-example}
\end{listing}

\noindent These operators were used in \hyperref[section:nexmark]{NEXMark}.

\subsubsection{Reduce}
Reducing data in some form is a very frequent operation in dataflows. This operator offers multiple variants of reduction for ease-of-use. A generic \code{reduce} that requires a key extractor, an initial value, a reductor, and a completor. The key extractor decides the grouping of the data, and the reductor is responsible for computing the intermediate reduction result for every record that arrives. Once an epoch is complete, the completor is invoked in order to compute the final output data from the intermediate reduction, the count of records, and the key for this batch of records. The variants \code{reduce_by}, \code{average_by}, \code{maximize_by}, and \code{minimize_by} build on top of this to provide more convenient access to reduction.

\begin{listing}[H]
\begin{minted}{rust}
fn reduce(&self, key_extractor, initial, reductor, completor) -> Stream{
    let mut epochs = HashMap::new();

    self.unary_notify(move |input, output, notificator| {
        input.for_each(|time, data| {
            let window = epochs.entry(time).or_insert_with(HashMap::new);
            data.for_each(|dat|{
                let key = key(&dat);
                let (v, c) = window.remove(&key).or((initial, 0));
                let value = reductor(dat, v);
                window.insert(key, (value, c+1));
            });
            notificator.notify_at(time);
        });
        notificator.for_each(|time, _, _| {
            if let Some(mut window) = epochs.remove(time) {
                output.session(time).give_iterator(
                    window.map(|(k, (v, c))| completor(k, v, c)));
            }
        });
    })
}
\end{minted}
  \caption{Simplified code for the general reduce operator.}
  \label{lst:reduce}
\end{listing}

\begin{listing}[H]
\begin{minted}{rust}
products.reduce(|_| 0, Product::new(0), |product, highest| {
    if highest.price < product.price { product } else { highest }
})
\end{minted}
\caption{A reduction example to find the product with the highest price.}
\label{lst:reduce-example}
\end{listing}

Finally, a separate \code{reduce_to} operator does not key data and instead reduces all data within the epoch to a single record.

\begin{listing}[H]
\begin{minted}{rust}
fn reduce_to(&self, initial_value, reductor) -> Stream {
    let mut epochs = HashMap::new();
    
    self.unary_notify(move |input, output, notificator| {
        input.for_each(|time, data| {
            let mut reduced = epochs.remove(time).or(initial_value);
            while let Some(dat) = data.pop() {
                reduced = reductor(dat, reduced);
            };
            epochs.insert(time, reduced);
            notificator.notify_at(time);
        });
        notificator.for_each(|time, _, _| {
            if let Some(reduced) = epochs.remove(time) {
                output.session(time).give(reduced);
            }
        });
    })
}
\end{minted}
  \caption{Simplified code for the reduce to operator.}
  \label{lst:reduce-to}
\end{listing}

This operator can be useful when trying to compare against a common value among all records.

\begin{listing}[H]
\begin{minted}{rust}
records.reduce_to(0, |_, c| c+1)
\end{minted}
\caption{An example showing how to count the number of records in an epoch.}
\label{lst:reduce-to-example}
\end{listing}

\noindent These operators were used in \hyperref[section:ysb]{YSB}, \hyperref[section:hibench]{HiBench}, and \hyperref[section:nexmark]{NEXMark}.

\subsubsection{RollingCount}
The \code{rolling_count} operator is similar to a reductor, but has a few distinct differences. First, it emits an output record for every input record it sees, rather than only once per epoch. Second, it keeps the count across epochs, rather than resetting for each epoch. Finally, it can only count records, rather than performing arbitrary reduction operations.

\begin{listing}[H]
\begin{minted}{rust}
fn rolling_count(&self, key_extractor, counter) -> Stream{
    let mut counts = HashMap::new();
    
    self.unary_stream(move |input, output| {
        input.for_each(|time, data| {
            output.session(time).give_iterator(data.map(|x|{
                let key = key(&x);
                let count = counts.get(&key).unwrap_or(0)+1;
                counts.insert(key.clone(), count);
                counter(x, count)
            }));
        });
    })
}
\end{minted}
  \caption{Simplified code for the rolling count operator.}
  \label{lst:rolling-count}
\end{listing}

The most trivial use-case is the classic word count benchmark.

\begin{listing}[H]
\begin{minted}{rust}
words.rolling_count(|word| word.clone(), |word, count| (word, count))
\end{minted}
\caption{A basic word count example using the rolling-count operator.}
\label{lst:rolling-count-example}
\end{listing}

\noindent This operator was used in \hyperref[section:hibench]{HiBench}.

\subsubsection{Window}
The window operator batches records together into windows. Windows can be sliding or hopping, and can be of arbitrary size, although they are limited in their granularity by epochs. This means that the epochs need to be correlated to a unit that the user would like to window by. When the window is full and the frontier reaches a slide, the window operator sends out a copy of all records within the window.

\begin{listing}[H]
\begin{minted}{rust}
fn window(&self, size, slide, time) -> Stream
    let mut parts = HashMap::new();
    self.unary_notify(move |input, output, notificator| {
        input.for_each(|cap, data| {
            data.drain(..).for_each(|data|{
                let time = time(cap.time(), &data);
                // Push the data onto a partial window.
                let part = parts.entry(time).or_insert_with(Vec::new);
                part.push(data);
                // Calculate the next epochs on which this partial window
                // would be output in a slide, then notify on those times
                for i in 0..size/slide {
                    let target = if time < size { size-1
                    } else { size-1+((time-size)/slide+1+i)*slide };
                    notificator.notify_at(cap.delayed(target));
                }
            });
        });
        
        notificator.for_each(|cap, _, _| {
            let end = cap.time();
            let mut time = end+1-size;
            let slide_end = time+slide;
            let mut window = Vec::new();
            // Compute full window from partials. First gather parts
            // that would fall out of the window and remove them.
            while time < slide_end {
                if let Some(mut part) = parts.remove(time) {
                    window.append(&mut part);
                }
                time += 1;
            }
            // Then gather and clone parts that will still be relevant
            // later.
            while time <= end {
                if let Some(part) = parts.get(time) {
                    part.iter().for_each(|e| window.push(e.clone()));
                }
                time += 1;
            }
            // Finally output the full window.
            output.session(&cap).give_iterator(window.drain(..));
        });
    })
}
\end{minted}
  \caption{Simplified code for the general window operator.}
  \label{lst:epoch-window}
\end{listing}

The generic window operator has some overhead. Thus there is a specific operator for tumbling windows which should be a lot more efficient.

\begin{listing}[H]
\begin{minted}{rust}
fn tumbling_window(&self, size) -> Stream{
    let mut windows = HashMap::new();
    
    self.unary_notify(move |input, output, notificator| {
        let size = size.clone();
        input.for_each(|cap, data| {
            // Round the time up to the next window.
            let wtime = (cap.time() / size + 1) * size;
            // Now act as if we were on that window's time.
            notificator.notify_at(cap.delayed(wtime));
            let window = windows.entry(wtime).or_insert_with(Vec::new);
            data.drain(..).for_each(|data|{
                window.push(data);
            });
        });
        
        notificator.for_each(|cap, _, _| {
            if let Some(mut window) = windows.remove(cap.time()) {
                output.session(&cap).give_iterator(window.drain(..));
            }
        });
    })
}
\end{minted}
  \caption{Simplified code for the tumbling window operator.}
  \label{lst:tumbling-window}
\end{listing}

A typical example of a tumbling window usage is to batch events together into well-defined intervals.

\begin{listing}[H]
\begin{minted}{rust}
frames.tumbling_window(60)
\end{minted}
\caption{An example of a tumbling window, batching frames into intervals of minutes, assuming an epoch represents one second.}
\label{lst:tumbling-window-example}
\end{listing}

\noindent These operators were used in \hyperref[section:ysb]{YSB}, \hyperref[section:hibench]{HiBench}, and \hyperref[section:nexmark]{NEXMark}.

\subsubsection{Session}
The session operator is similar to a window: it batches records, but instead of using a regular interval, a session is only completed after a certain timeout has been reached. As an example, a session with a timeout of 10 seconds would only be complete if there were no records for 10 seconds on the stream. Before this timeout is reached, all incoming records are gathered into the current session.

\begin{listing}[H]
\begin{minted}{rust}
fn session(&self, timeout, sessioner) -> Stream{
    let mut sessions = HashMap::new();
    
    self.unary_notify(move |input, output, notificator| {
        input.for_each(|cap, data| {
            for data in data.drain(..){
                let (s, t) = key(&data);
                notificator.notify_at(cap.delayed(t + timeout));
                let session = sessions
                    .entry(t).or_insert_with(HashMap::new)
                    .entry(s).or_insert_with(Vec::new);
                session.push(data);
            };
        });
        
        notificator.for_each(|cap, _, _| {
            // For each session at the original time we need to check if
            // it has expired, or if we need to delay.
            let otime = cap.time() - timeout;
            let mut expired = sessions.remove(&otime).or(HashMap::new);
            expired.drain().for_each(|(s, mut d)| {
                // Now we check backwards from the current epoch.
                let mut found = false;
                for i in 0..timeout {
                    let t = cap.time() - i;
                    if let Some(session) = sessions.get_mut(&t) {
                        if let Some(data) = session.get_mut(&s) {
                            // If we find data within the timeout, delay
                            // our data to that later time. If that time
                            // does not happen to be final either, both
                            //  this and that data will get moved ahead
                            //even further automatically.
                            data.append(&mut d);
                            found = true;
                            break;
                        }
                    }
                }
                if !found {
                    // If we don't find a any data within the timeout,
                    // the session is full and we can output it.
                    output.session(&cap).give((s, d));
                }
            });
        });
    })
}
\end{minted}
  \caption{Simplified code for the session operator.}
  \label{lst:session}
\end{listing}

Sessions can be useful to track intervals of activity, for instance to try and estimate periods of time during which a user is actively visiting a site.

\begin{listing}[H]
\begin{minted}{rust}
tweets.session(3600, |tweet| (tweet.author, tweet.time / 1000))
\end{minted}
\caption{An example of a session to determine batches of tweets during which the user is active.}
\label{lst:session-example}
\end{listing}

\noindent This operator was used in \hyperref[section:nexmark]{NEXMark}.

\subsubsection{Partition}
The partitioning operator transforms the data stream into windows of a fixed number of records, each keyed by a property.

\begin{listing}[H]
\begin{minted}{rust}
fn partition(&self, size, key) -> Stream{
    let mut partitions = HashMap::new();

    self.unary_stream(move |input, output| {
        input.for_each(|time, data| {
            data.for_each(|dat| {
                let key = key(&dat);
                let mut partition = partitions.remove(&key)
                                        .or(|| Vec::with_capacity(size));
                partition.push(dat);
                if partition.len() == size {
                    output.session(time).give(partition);
                } else {
                    partitions.insert(key, partition);
                }
            });
        });
    })
}
\end{minted}
  \caption{Simplified code for the partitioning operator.}
  \label{lst:reduce-to}
\end{listing}

Partitioning is useful when we need to be certain about the number of events present in the stream at any particular point.

\begin{listing}[H]
\begin{minted}{rust}
frames.partition(30, |frame| frame.animation)
\end{minted}
\caption{This creates 30 frame (one second) animation batches from a stream of frames.}
\label{lst:partition-example}
\end{listing}

\noindent This operator was used in \hyperref[section:nexmark]{NEXMark}.

%%% Local Variables:
%%% mode: latex
%%% TeX-master: "../thesis"
%%% End:
