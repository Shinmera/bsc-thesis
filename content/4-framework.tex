\section{Benchmark Framework}


\subsection{Additional Operators}
In order to ease the implementation of the benchmarks and improve the usability of the Timely system we introduced a number of additional operators. We will outline and discuss these operators here shortly.

\paragraph{FilterMap}
This operator performs a filter followed by a map in one go. This is useful when the data stream contains mixed event types and only a single type of event is required. The operator expects a closure which can choose to either filter events by returning \code{None}, or map them by returning \code{Some(..)} with the output data.

\paragraph{Join}
This operator offers two forms of joins that merge two separate streams of data into one. The first is an epoch based join, meaning data is only matched up between the two streams within a single epoch. If no match is found for either stream, the data is discarded. The second is a left join that keeps the left-hand stream's data around indefinitely, continuously joining it with data from the right-hand stream whenever the keys match.

\paragraph{Reduce}
Reducing data in some form is a very frequent operation in dataflows. This operator offers multiple variants of reduction for ease-of-use. A generic \code{reduce} that requires a key extractor, an initial value, a reductor, and a completor. The key extractor decides the grouping of the data, and the reductor is responsible for computing the intermediate reduction result for every record that arrives. Once an epoch is complete, the completor is invoked in order to compute the final output data from the intermediate reduction, the count of records, and the key for this batch of records. The variants \code{reduce_by} and \code{average_by} build on top of this to provide more convenient access to reduction. Finally, a separate \code{reduce_to} does not key data and instead reduces all data within the epoch to a single record.

\paragraph{RollingCount}
The \code{rolling_count} operator is similar to a reductor, but has a few distinct differences. First, it emits an output record for every input record it sees, rather than only once per epoch. Second, it keeps the count across epochs, rather than resetting for each epoch. Finally, it can only count records, rather than performing arbitrary reduction operations.

\paragraph{Window}
The window operator batches records together into windows. Windows can be sliding or hopping, and can be of arbitrary size, although they are limited in their granularity by epochs. This means that a window cannot be smaller than a single epoch. When the window is full and the frontier reaches a slide, the window operator sends out a copy of all records within the window.

%%% Local Variables:
%%% mode: latex
%%% TeX-master: "../thesis"
%%% End:
