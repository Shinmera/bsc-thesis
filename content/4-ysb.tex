\section{Yahoo Streaming Benchmark (YSB)\cite{ysb}}
The Yahoo Streaming Benchmark is a single dataflow benchmark created by Yahoo in 2015. Of the three benchmark suites implemented in this thesis, it is the one most widely used in the industry. The original implementation includes support for Storm, Spark, Flink, and Apex. \\

\imagefigure[ysb]{images/ysb-graphs.pdf}{A graph of the dataflow described by YSB.}

The dataflow used in the benchmark is illustrated in \autoref{figure:ysb}. Its purpose is to count ad hits for each ad campaign. Events arrive from Kafka in JSON string format, where each event is a flat object with the following fields:

\begin{itemize}
\item \code{user_id} A UUID identifying the user that caused the event.
\item \code{page_id} A UUID identifying the page on which the event occurred.
\item \code{ad_id} A UUID for the specific advertisement that was interacted with.
\item \code{ad_type} A string, one of ``banner'', ``modal'', ``sponsored-search'', ``mail'', and ``mobile''.
\item \code{event_type} A string, one of ``view'', ``click'', and ``purchase''.
\item \code{event_time} An integer timestamp in milliseconds of the time the event occurred.
\item \code{ip_address} A string of the user's IP address.
\end{itemize}

The dataflow proceeds as follows: the first operator parses the JSON string into an internal object. Irrelevant events are then filtered out, and only ones with an \code{event_type} of ``view'' are retained. Next, all fields except for \code{ad_id} and \code{event_time} are dropped. Then, a lookup in a table mapping \code{ad_id}s to \code{campaign_id}s is done to retrieve the relevant \code{campaign_id}. Yahoo describes this step as a join, which is inaccurate, as only one end of this ``join'' is streamed, whereas the other is present as a table stored in Redis. Next the events are put through a ten seconds large hopping window. The number of occurrences of each \code{campaign_id} within each window are finally counted and stored back into Redis.

\subsection{Implementation}

\begin{listing}[H]
  \inputminted[firstline=150,lastline=164]{rust}{benchmarks/src/ysb.rs}
  \caption{Dataflow implementation of the YSB benchmark.}
  \label{lst:ysb}
\end{listing}

\subsection{Evaluation}


%%% Local Variables:
%%% mode: latex
%%% TeX-master: "../thesis"
%%% TeX-engine: luatex
%%% TeX-command-extra-options: "-shell-escape"
%%% End:
