\documentclass[14pt,t]{beamer}
\usepackage{fontspec}
\usepackage{color}
\usepackage{xcolor}
\usepackage{minted}
\usepackage{pgfplots}
\usepackage{etoolbox}
\usepackage[group-separator={'}]{siunitx}

%% These fonts are non-free.
%% Comment out the lines if you don't have them.
\setmainfont{Equity Text A}
\setsansfont{Concourse T3}
\setmonofont{Triplicate T4}

\definecolor{bgcolor}{RGB}{20,25,28}
\definecolor{codecolor}{RGB}{249,38,114}
\hypersetup{colorlinks,linkcolor=,urlcolor=codecolor}
\setbeamercolor{background canvas}{bg=bgcolor}
\setbeamercolor{normal text}{fg=white}
\setbeamercolor{itemize item}{fg=lightgray}
\setbeamercolor{itemize subitem}{fg=lightgray}
\setbeamercolor{itemize subsubitem}{fg=lightgray}
\setbeamercolor{enumerate item}{fg=lightgray}
\setbeamertemplate{itemize items}[circle]
\setbeamertemplate{navigation symbols}{}
\usemintedstyle{monokai}
\newminted[rustcode]{rust}{fontsize=\footnotesize}

\def\code#1{{\color{codecolor}\texttt{#1}}}

\renewcommand{\theFancyVerbLine}{\color{darkgray}\large \oldstylenums{\arabic{FancyVerbLine}}}
\renewcommand{\title}[1]{
  {\huge #1} \vskip 0.4cm
}
\renewcommand{\subtitle}[1]{
  \vskip 0.3cm {\Large #1} \vskip 0.2cm %
}

\usetikzlibrary{
  pgfplots.colorbrewer,
}
\pgfplotsset{
  % define a `cycle list' for marker
  cycle list/.define={my marks}{
    every mark/.append style={solid,fill=\pgfkeysvalueof{/pgfplots/mark list fill}},mark=*\\
    every mark/.append style={solid,fill=\pgfkeysvalueof{/pgfplots/mark list fill}},mark=square*\\
    every mark/.append style={solid,fill=\pgfkeysvalueof{/pgfplots/mark list fill}},mark=triangle*\\
    every mark/.append style={solid,fill=\pgfkeysvalueof{/pgfplots/mark list fill}},mark=diamond*\\
  },
}

\pgfplotsset{tikzDefaults/.style=
  {legend pos=outer north east,
    legend style={fill=none},
    ymajorgrids=true,
    grid style=dashed,
    cycle list/Spectral-11,
    cycle multiindex* list={
      Spectral-11
      \nextlist
      my marks
      \nextlist
      [3 of]linestyles
      \nextlist
      very thick
      \nextlist},
  }}

\makeatletter
\newcommand{\latencyplot}[3]{%
  \begin{tikzpicture}
    \begin{axis}[
      tikzDefaults,
      title={Median Latency (#1 workers)},
      ylabel={Rate [e/s]},
      xlabel={Latency [s]},
      xmode=log, ymode=log,
      x tick label style={/pgf/number format/fixed},
      ]
      
      \forcsvlist{\latencyplot@item{#1}}{#2}
      \ifthenelse{ \equal{#3}{} }{}{\legend{#3}}
    \end{axis}
  \end{tikzpicture}}

\newcommand{\latencyplot@item}[2]{%
  \addplot table[y index=0, x index=#2] {../data/latency-#1.csv};%
}

\newcommand{\scalingplot}[3]{%
  \begin{tikzpicture}
    \begin{axis}[
      tikzDefaults,
      title={Scaling (\num{#1} e/s)},
      xlabel={Workers [n]},
      ylabel={Latency [s]},
      xmode=log,
      log basis x={2},
      scaled y ticks=false,
      y tick label style={/pgf/number format/fixed},
      xtick={1, 2, 4, 8, 16, 32},
      ]
      
      \forcsvlist{\scalingplot@item{#1}}{#2}
      \ifthenelse{ \equal{#3}{} }{}{\legend{#3}}
    \end{axis}
  \end{tikzpicture}}

\newcommand{\scalingplot@item}[2]{%
  \addplot table[x index=0, y index=#2] {../data/scaling-#1.csv};%
}

\newcommand{\cdfplot}[4]{%
  \begin{tikzpicture}
    \begin{axis}[
      tikzDefaults,
      title={CDF (#1 workers, \num{#2} e/s)},
      xlabel={Latency [s]},
      ylabel={CDF [\%]},
      xmode=log,
      ]
      
      \forcsvlist{\cdfplot@item{#1}{#2}}{#3}
      \ifthenelse{ \equal{#4}{} }{}{\legend{#4}}
    \end{axis}
  \end{tikzpicture}}

\newcommand{\cdfplot@item}[3]{%
  \addplot table[skip first n=1, y index=0, x index=#3] {../data/cdf-#1-#2.csv};%
}

\newcommand{\ccdfplot}[4]{%
  \begin{tikzpicture}
    \begin{axis}[
      tikzDefaults,
      title={CCDF (#1 workers, #2 e/s)},
      xlabel={Latency [s]},
      ylabel={CCDF [\%]},
      xmode=log, x dir=reverse
      ]
      
      \forcsvlist{\cdfplot@item{#1}{#2}}{#3}
      \ifthenelse{ \equal{#4}{} }{}{\legend{#4}}
    \end{axis}
  \end{tikzpicture}}

\makeatother

\begin{document}
\begin{frame}
  \begin{center}
    \vspace{2cm}
    {\LARGE Implementation of a Benchmark Suite for Strymon} \\
    \vspace{0.5cm}
    Nicolas Hafner \\
    \vspace*{2.9cm}
    \hspace*{-1cm}\colorbox{white}{\begin{minipage}{\pagewidth}
        \begin{center}
          \includegraphics[width=5cm]{../systems-cover/ethlogo_black.pdf}
          \hspace{1cm}
          \includegraphics[width=5cm]{../systems-cover/inf-logo.pdf}
        \end{center}
      \end{minipage}} \\
  \end{center}
\end{frame}

\begin{frame}
  \title{Motivation}
  \begin{itemize}
  \item Performance is important for systems
  \item Measuring performance can be tough
  \item Would like to compare to competitors
  \item Industry benchmarks need to be evaluated
  \end{itemize}
\end{frame}

\begin{frame}
  \title{Other Works}
  \begin{itemize}
  \item Investigated 9 papers for other systems
  \item Often very simple benchmarks like Word Count
  \item Code and data often not published
  \item No paper used a standardised benchmark
  \end{itemize}
\end{frame}

\begin{frame}
  \title{Timely}
  \begin{itemize}
  \item Streaming processing based on data parallelism
  \item Computations expressed via data flow graphs
  \item Each physical worker runs the entire data flow
  \item Physical exchange of data handled by Timely
  \item Data is separated into chunks called epochs
  \item Computation progress tracked by epoch frontiers
  \end{itemize}
\end{frame}

\begin{frame}
  \title{Benchmarks}
  \begin{itemize}
  \item We implemented three benchmarks:
  \end{itemize}
  \begin{enumerate}
  \item Intel's HiBench
  \item Yahoo's Streaming Benchmark
  \item Apache Beam's NEXMark
  \end{enumerate}
\end{frame}

\begin{frame}
  \title{HiBench}
  \begin{itemize}
  \item General Big Data benchmark
  \item Includes a section for streaming systems
  \item Only four latency tests:
  \end{itemize}
  \begin{enumerate}
  \item Identity \hfill\raisebox{-.2\height}{\includegraphics[height=0.75cm]{hib-1.png}}
  \item Repartition \hfill\raisebox{-.2\height}{\includegraphics[height=0.75cm]{hib-2.png}}
  \item Word Count \hfill\raisebox{-.2\height}{\includegraphics[height=0.75cm]{hib-3.png}}
  \item Window Reduce \hfill\raisebox{-.2\height}{\includegraphics[height=0.75cm]{hib-4.png}}
  \end{enumerate}
\end{frame}

\begin{frame}
  \title{YSB}
  \begin{itemize}
  \item Initially developed internally at Yahoo:
  \item Count ad views for ad campaigns
  \item Original setup requires Redis table lookup
  \item Only one, relatively simple data flow:
  \end{itemize}
  \includegraphics[height=0.75cm]{ysb.png}
\end{frame}

\begin{frame}
  \title{NEXMark}
  \begin{itemize}
  \item Based on a paper by Tucker et al.
  \item An adaptation of XMark for streaming
  \item Implements an ``auctioning system''
  \item 13 data flows in total
  \item Uses filter, map, reduce, join, window, session, partition
  \item Dataflows for Query 5 and 8:
  \end{itemize}
  \includegraphics[height=1.5cm]{nex-5.png}
  \hfill
  \includegraphics[height=1.5cm]{nex-8.png}
\end{frame}

\begin{frame}
  \title{Evaluation}
  \begin{itemize}
  \item Run on sgs-r815-03 (32 cores, 2.4GHz)
  \item Measured closed-loop per-epoch latency
  \item Data generated directly in memory
  \item Timely performs very well!
  \end{itemize}
\end{frame}

\begin{frame}
  \title{NEXMark Latency Scaling}
  \latencyplot{32}{6,7,8,9,10,11,12,13,14,15,16}{Q0, Q1, Q2, Q3, Q4, Q5, Q6, Q7, Q8, Q9, Q11}
\end{frame}

\begin{frame}
  \title{NEXMark Worker Scaling}
  \scalingplot{10000000}{6,7,8,9,10,11,12,13,14,15,16}{Q0, Q1, Q2, Q3, Q4, Q5, Q6, Q7, Q8, Q9, Q11}
\end{frame}

\begin{frame}
  \title{NEXMark CDF}
  \cdfplot{32}{10000000}{6,7,8,9,10,11,12,13,14,15,16}{Q0, Q1, Q2, Q3, Q4, Q5, Q6, Q7, Q8, Q9, Q11}
\end{frame}

\begin{frame}
  \title{Conclusion}
  \begin{itemize}
  \item Timely competes very well with other systems
  \item Existing benchmarks leave things to be desired
  \item New benchmarks need:
    \begin{itemize}
    \item Abstract model definitions for data flows
    \item Various short and long data flows
    \item Deterministically generated workloads
    \item Correctness verification tools
    \end{itemize}
  \end{itemize}
\end{frame}

\begin{frame}
  \title{Thanks}
  Thanks to John Liagouris, Frank McSherry, and the whole DCModel team!
  
  \vspace*{5.3cm}
  \hspace*{-1cm}\colorbox{white}{\begin{minipage}{\pagewidth}
      \begin{center}
        \includegraphics[width=5cm]{../systems-cover/ethlogo_black.pdf}
        \hspace{1cm}
        \includegraphics[width=5cm]{../systems-cover/inf-logo.pdf}
      \end{center}
    \end{minipage}} \\
\end{frame}
\end{document}

%%% Local Variables:
%%% mode: latex
%%% TeX-master: t
%%% TeX-engine: luatex
%%% TeX-command-extra-options: "-shell-escape"
%%% End:
